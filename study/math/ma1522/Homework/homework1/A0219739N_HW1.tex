\documentclass{article}
\usepackage{enumitem}
\usepackage{amsmath}
\usepackage{amssymb}
\usepackage{mathtools}
\usepackage[a4paper, total={6in, 8in}]{geometry}

% augmented matrix
\newenvironment{amatrix}[1]{%
  \left(\begin{array}{@{}*{#1}{c}|c@{}}
}{%
  \end{array}\right)
}

% row operation arrow
\newcommand{\ro}[1]{%
  \xrightarrow{\mathmakebox{#1}}%
}

% stretch array 
\makeatletter
\renewcommand*\env@matrix[1][\arraystretch]{%
  \edef\arraystretch{#1}%
  \hskip -\arraycolsep
  \let\@ifnextchar\new@ifnextchar
  \array{*\c@MaxMatrixCols c}}
\makeatother

\title{Homework 1 - Submission}
\author{A0219739N - Le Van Minh}
\date{\today}

\begin{document}
\maketitle
\begin{enumerate}
  % question 1
  \item \begin{enumerate}[label={(\roman*)}]
    \item 
      \[
        \begin{amatrix}{3}
            1 & 1 & 3 - a & 2 \\
            3 & 4 & 2 & b \\
            2 & 3 & -1 & 1
        \end{amatrix}
      \]
    \item
      \begin{alignat*}{2}
        \begin{amatrix}{3}
            1 & 1 & 3 - a & 2 \\
            3 & 4 & 2 & b \\
            2 & 3 & -1 & 1
        \end{amatrix}
        \ro{R_2 - 3R_1}
        \begin{amatrix}{3}
            1 & 1 & 3 - a & 2 \\
            0 & 1 & 3a-7 & b-6 \\
            2 & 3 & -1 & 1
        \end{amatrix}
        \ro{R_3 - 2R_1}
        \begin{amatrix}{3}
            1 & 1 & 3 - a & 2 \\
            0 & 1 & 3a-7 & b-6 \\
            0 & 1 & 2a - 7 & -3
        \end{amatrix}
        \\
        \ro{R_3 - R_1}
        \begin{amatrix}{3}
            1 & 1 & 3 - a & 2 \\
            0 & 1 & 3a-7 & b-6 \\
            0 & 0 & -a & 3 - b
        \end{amatrix}
        \ro{R_1 - R_2}
        \begin{amatrix}{3}
            1 & 0 & 10 - 4a & 8 - b \\
            0 & 1 & 3a-7 & b-6 \\
            0 & 0 & -a & 3 - b
        \end{amatrix}
      \end{alignat*}
      For the system to have no solution, the last row can be made so that all cells on the LHS are zeros, but the cell in the RHS is non-zero. One such instance is:
      \begin{align*}
        \begin{pmatrix}
          -a \\ 3 - b
        \end{pmatrix}
        &= 
        \begin{pmatrix}
          0 \\ 3
        \end{pmatrix}
        \\
        \Rightarrow
        \begin{pmatrix}
          a \\ b
        \end{pmatrix}
        &= 
        \begin{pmatrix}
          0 \\ 0
        \end{pmatrix}
      \end{align*}
    \item We can get a system with infinite solutions by making the last row a zero row.
      \begin{align*}
        \begin{pmatrix}
          -a \\ 3 - b
        \end{pmatrix}
        &= 
        \begin{pmatrix}
          0 \\ 0
        \end{pmatrix}
        \\
        \Rightarrow
        \begin{pmatrix}
          a \\ b
        \end{pmatrix}
        &= 
        \begin{pmatrix}
          0 \\ 3
        \end{pmatrix}
      \end{align*}
      The augmented matrix became:
      \[
        \begin{amatrix}{3}
          1 & 0 & 10 & 5 \\
          0 & 1 & -7 & -3 \\
          0 & 0 & 0 & 0 
        \end{amatrix}
      \]
      Let $z = t$:
      \[
        \begin{cases}
          x + 10z = 5 \\
          y - 7z = -3 \\
          z = t
        \end{cases}
        \Rightarrow
        \begin{cases}
          x = 5 - 10t \\
          y = -3 + 7t \\
          z = t
        \end{cases}
        \Rightarrow
        \begin{pmatrix}
          x \\ y \\ z 
        \end{pmatrix}
        =
        \begin{pmatrix}
          5 \\ -3 \\ 0
        \end{pmatrix} +
        t
        \begin{pmatrix}
          -10 \\ 7 \\ 1
        \end{pmatrix}
      \]
    \item The system has unique solution when $-a \neq 0$. One possible parameter is:
      \[
        \begin{pmatrix}
          -a \\ 3 - b
        \end{pmatrix}
        = 
        \begin{pmatrix}
          1 \\ 0
        \end{pmatrix}
        \Rightarrow
        \begin{pmatrix}
          a \\ b
        \end{pmatrix}
        = 
        \begin{pmatrix}
          -1 \\ 3
        \end{pmatrix}
      \]
      The last row translates to:
      \[
        z = 0
      \]
      which is the value of $z$ of the unique solution.
  \end{enumerate}

  % question 2
\item \begin{enumerate}[label={(\roman*)}]
    \item 
      \begin{alignat*}{2}
        \begin{pmatrix}
          1 & 0 & 2 \\
          2 & - 1 & 3 \\
          4 & 1 & 6
        \end{pmatrix}
        \ro{R_2 - 2R_1}
        \begin{pmatrix}
          1 & 0 & 2 \\
          0 & -1 & -1 \\
          4 & 1 & 6
        \end{pmatrix}
        \ro{R_3 - 4R_1}
        \begin{pmatrix}
          1 & 0 & 2 \\
          0 & -1 & -1 \\
          0 & 1 & -2
        \end{pmatrix}
        \\
        \ro{-R_2}
        \begin{pmatrix}
          1 & 0 & 2 \\
          0 & 1 & 1 \\
          0 & 1 & -2
        \end{pmatrix}
        \ro{R_3 - R_2}
        \begin{pmatrix}
          1 & 0 & 2 \\
          0 & 1 & 1 \\
          0 & 0 & -3
        \end{pmatrix}
        \ro{-\frac{1}{3}R_3}
        \begin{pmatrix}
          1 & 0 & 2 \\
          0 & 1 & 1 \\
          0 & 0 & 1
        \end{pmatrix}
      \end{alignat*}
      Then:
      \[
        B = 
        \begin{pmatrix}
          1 & 0 & 2 \\
          0 & 1 & 1 \\
          0 & 0 & 1
        \end{pmatrix}
      \]
    \item The reduced echelon form of B is:
      \begin{alignat*}{2}
        \begin{pmatrix}
          1 & 0 & 2 \\
          0 & 1 & 1 \\
          0 & 0 & 1
        \end{pmatrix}
        \ro{R_1 - 2R_3}
        \begin{pmatrix}
          1 & 0 & 0 \\
          0 & 1 & 1 \\
          0 & 0 & 1
        \end{pmatrix}
        \ro{R_2 - R_3}
        \begin{pmatrix}
          1 & 0 & 0 \\
          0 & 1 & 0 \\
          0 & 0 & 1
        \end{pmatrix}
      \end{alignat*}
      Then:
      \[
        R = 
        \begin{pmatrix}
          1 & 0 & 0 \\
          0 & 1 & 0 \\
          0 & 0 & 1
        \end{pmatrix}
      \]
    \item 
      \begin{alignat*}{2}
        R = 
        \begin{pmatrix}
          1 & 0 & 0 \\
          0 & 1 & 0 \\
          0 & 0 & 1
        \end{pmatrix}
        =
        \begin{pmatrix}
          1 & 0 & 0 \\
          0 & 1 & -1 \\
          0 & 0 & 1
        \end{pmatrix}
        \begin{pmatrix}
          1 & 0 & -2 \\
          0 & 1 & 0 \\
          0 & 0 & 1
        \end{pmatrix}
        \begin{pmatrix}
          1 & 0 & 0 \\
          0 & 1 & 0 \\
          0 & 0 & -\frac{1}{3}
        \end{pmatrix}
        \begin{pmatrix}
          1 & 0 & 0 \\
          0 & 1 & 0 \\
          0 & -1 & 1
        \end{pmatrix}\\ 
        \begin{pmatrix}
          1 & 0 & 0 \\
          0 & -1 & 0 \\
          0 & 0 & 1
        \end{pmatrix}
        \begin{pmatrix}
          1 & 0 & 0 \\
          0 & 1 & 0 \\
          -4 & 0 & 1
        \end{pmatrix}
        \begin{pmatrix}
          1 & 0 & 0 \\
          -2 & 1 & 0 \\
          0 & 0 & 1
        \end{pmatrix}
        A
      \end{alignat*}
    \item From (i), (ii), $A$ can be transformed into $R = I_3$ through a series of elementary row operation, i.e. $A$ is row equivalent to $I_3$. Therefore, by Theorem (7), the series of elementary row operation that transform $A$ into $I_3$ also transforms $I_3$ into $A^{-1}$, and $A^{-1}$ is the product of those respective elementary matrices.
    \item 
      \begin{alignat*}{2}
        A^{-1} &= 
        \begin{pmatrix}
          1 & 0 & 0 \\
          0 & 1 & -1 \\
          0 & 0 & 1
        \end{pmatrix}
        \begin{pmatrix}
          1 & 0 & -2 \\
          0 & 1 & 0 \\
          0 & 0 & 1
        \end{pmatrix}
        \begin{pmatrix}
          1 & 0 & 0 \\
          0 & 1 & 0 \\
          0 & 0 & -\frac{1}{3}
        \end{pmatrix}
        \begin{pmatrix}
          1 & 0 & 0 \\
          0 & 1 & 0 \\
          0 & -1 & 1
        \end{pmatrix}\\ 
        &\quad \begin{pmatrix}
          1 & 0 & 0 \\
          0 & -1 & 0 \\
          0 & 0 & 1
        \end{pmatrix}
        \begin{pmatrix}
          1 & 0 & 0 \\
          0 & 1 & 0 \\
          -4 & 0 & 1
        \end{pmatrix}
        \begin{pmatrix}
          1 & 0 & 0 \\
          -2 & 1 & 0 \\
          0 & 0 & 1
        \end{pmatrix}
        I_3 \\
        &= 
        \begin{pmatrix}[1.3]
          -3 & \frac{2}{3} & \frac{2}{3} \\
          0 & -\frac{2}{3} & \frac{1}{3} \\
          2 & - \frac{1}{3} & - \frac{1}{3}
        \end{pmatrix}
      \end{alignat*}
    \end{enumerate}
    % question 3
  \item
    Consider the (1,1) cell in each of these products:
    \begin{align*}
      A_{21} B_{11} &= \begin{pmatrix}
        a_{31} b_{11} + a_{32} b_{21} & * \\
        * & *
      \end{pmatrix} \\
      A_{22} B_{21} &= \begin{pmatrix}
        a_{33} b_{31} + a_{34} b_{41} & * \\
        * & *
      \end{pmatrix} \\
      A_{23} B_{31} &= \begin{pmatrix}
        a_{35} b_{51} + a_{36} b_{61} & * \\
        * & *
      \end{pmatrix} \\
    \end{align*}
    \[
      (a_{31} b_{11} + a_{32} b_{12}) + (a_{33} b_{13} + a_{34} b_{14}) + (a_{35} b_{15} + a_{36} b_{16}) = \sum_{1 \le r \le 6} a_{3r} b_{r1}
    \]
    Similarly, we can obtain the values of the other cells:
    \begin{align*}
      A_{21} B_{11} + A_{22} B_{21} + A_{23} B_{31}
      &= \begin{pmatrix}
        \sum_{1 \le r \le 6} a_{3r} b_{r1} & \sum_{1 \le r \le 6} a_{3r} b_{r2} \\
        \sum_{1 \le r \le 6} a_{4r} b_{r1} & \sum_{1 \le r \le 6} a_{4r} b_{r2}
      \end{pmatrix} 
      &= \begin{pmatrix}
        c_{31} & c_{32} \\
        c_{41} & c_{42}
      \end{pmatrix}
      &= C_{21}
    \end{align*}
\end{enumerate}
\end{document}
