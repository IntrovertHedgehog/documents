\documentclass{article}
\usepackage{enumitem}
\usepackage{amsmath}
\usepackage[a4paper, total={6in, 8in}]{geometry}

\title{Assignment 2 - Submission}
\author{A0219739N - Le Van Minh}
\date{\today}

\begin{document}
\maketitle
\begin{enumerate}[leftmargin=\labelsep]
  \item 
    \begin{tabular}{l|l}
      \hline
      \multicolumn{1}{c|}{\textbf{Maturity}} & 
      \multicolumn{1}{c}{\textbf{YTM}} \\
      \hline
      1 & 4.7\% \\
      2 & 4.8\% \\
      3 & 5.0\% \\
      \hline
    \end{tabular}
  \item \begin{enumerate}
    \item 
      \[
        1135.9 = \frac{50}{r} \times \left(1 - \frac{1}{(1+r)^{20}} \right) + \frac{1000}{(1+r)^{20}} 
      \]
      \begin{align*}
        r & = 4\% \\
        YTM = r \times 2 &= 8\%
      \end{align*}
      
    \item 
      \begin{align*}
        r & = APR \div 2 = 5\% \\
        P & = \frac{50}{r} \times \left(1 - \frac{1}{(1+r)^{20}} \right) + \frac{1000}{(1+r)^{20}} \\
        & = 1000
      \end{align*}
  \end{enumerate}

  \item Supposed the bond face value is \$1000, the bond price is:
    \begin{align*}
      P &= \frac{50}{1+YTM_1} + \frac{50}{(1+YTM_2)^2} + \frac{1050}{(1+YTM_3)^3} \\
      &= 1000.3 \\
      &= \frac{50}{r} \times \left(1 - \frac{1}{(1+r)^{3}} \right) + \frac{1000}{(1+r)^{3}} \\
    \end{align*}
    \[
      r = YTM = 5 \%
    \]

  \item let $r_i$ be $i$-year risk free interest rate. With the one-year bond $r = YTM \div 2 = 3\%$:
    \begin{gather*}
      P = \displaystyle\sum_{i=1}^{2} \frac{4\% \times FV}{(1+3\%)^i} + \frac{FV}{(1+3\%)^2} = \frac{4\% \times FV}{1 + 2\%} + \frac{(4\% + 1) \times FV}{1 + r_1} \\
      r_1 = 6.1 \%
    \end{gather*}
    With the two-year bond $r = 5\%$:
    \begin{gather*}
      P = \displaystyle\sum_{i=1}^{2} \frac{10\% \times FV}{(1+5\%)^i} + \frac{FV}{(1+5\%)^2} = \frac{10\% \times FV}{1 + 6.1\%} + \frac{(10\% + 1) \times FV}{1 + r_2} \\
      r_2 = 10.1 \%
    \end{gather*}

  \item Suppose the bond promise to pay \$100 upon maturity, the expected return is $E(income) = 0.6 \times 100 + 0.4 \times 20 = 68$. With 10\% risk premium $r = 2\% + 10\% = 12\%$:
    \begin{gather*}
      P = \frac{68}{1 + 12\%} = 60.71 \\
      YTM = \frac{FV}{P} - 1 = 64.7 \%
    \end{gather*}
\end{enumerate}
\end{document}
